\documentclass{article}
%% ODER: format ==         = "\mathrel{==}"
%% ODER: format /=         = "\neq "
%
%
\makeatletter
\@ifundefined{lhs2tex.lhs2tex.sty.read}%
  {\@namedef{lhs2tex.lhs2tex.sty.read}{}%
   \newcommand\SkipToFmtEnd{}%
   \newcommand\EndFmtInput{}%
   \long\def\SkipToFmtEnd#1\EndFmtInput{}%
  }\SkipToFmtEnd

\newcommand\ReadOnlyOnce[1]{\@ifundefined{#1}{\@namedef{#1}{}}\SkipToFmtEnd}
\usepackage{amstext}
\usepackage{amssymb}
\usepackage{stmaryrd}
\DeclareFontFamily{OT1}{cmtex}{}
\DeclareFontShape{OT1}{cmtex}{m}{n}
  {<5><6><7><8>cmtex8
   <9>cmtex9
   <10><10.95><12><14.4><17.28><20.74><24.88>cmtex10}{}
\DeclareFontShape{OT1}{cmtex}{m}{it}
  {<-> ssub * cmtt/m/it}{}
\newcommand{\texfamily}{\fontfamily{cmtex}\selectfont}
\DeclareFontShape{OT1}{cmtt}{bx}{n}
  {<5><6><7><8>cmtt8
   <9>cmbtt9
   <10><10.95><12><14.4><17.28><20.74><24.88>cmbtt10}{}
\DeclareFontShape{OT1}{cmtex}{bx}{n}
  {<-> ssub * cmtt/bx/n}{}
\newcommand{\tex}[1]{\text{\texfamily#1}}	% NEU

\newcommand{\Sp}{\hskip.33334em\relax}


\newcommand{\Conid}[1]{\mathit{#1}}
\newcommand{\Varid}[1]{\mathit{#1}}
\newcommand{\anonymous}{\kern0.06em \vbox{\hrule\@width.5em}}
\newcommand{\plus}{\mathbin{+\!\!\!+}}
\newcommand{\bind}{\mathbin{>\!\!\!>\mkern-6.7mu=}}
\newcommand{\rbind}{\mathbin{=\mkern-6.7mu<\!\!\!<}}% suggested by Neil Mitchell
\newcommand{\sequ}{\mathbin{>\!\!\!>}}
\renewcommand{\leq}{\leqslant}
\renewcommand{\geq}{\geqslant}
\usepackage{polytable}

%mathindent has to be defined
\@ifundefined{mathindent}%
  {\newdimen\mathindent\mathindent\leftmargini}%
  {}%

\def\resethooks{%
  \global\let\SaveRestoreHook\empty
  \global\let\ColumnHook\empty}
\newcommand*{\savecolumns}[1][default]%
  {\g@addto@macro\SaveRestoreHook{\savecolumns[#1]}}
\newcommand*{\restorecolumns}[1][default]%
  {\g@addto@macro\SaveRestoreHook{\restorecolumns[#1]}}
\newcommand*{\aligncolumn}[2]%
  {\g@addto@macro\ColumnHook{\column{#1}{#2}}}

\resethooks

\newcommand{\onelinecommentchars}{\quad-{}- }
\newcommand{\commentbeginchars}{\enskip\{-}
\newcommand{\commentendchars}{-\}\enskip}

\newcommand{\visiblecomments}{%
  \let\onelinecomment=\onelinecommentchars
  \let\commentbegin=\commentbeginchars
  \let\commentend=\commentendchars}

\newcommand{\invisiblecomments}{%
  \let\onelinecomment=\empty
  \let\commentbegin=\empty
  \let\commentend=\empty}

\visiblecomments

\newlength{\blanklineskip}
\setlength{\blanklineskip}{0.66084ex}

\newcommand{\hsindent}[1]{\quad}% default is fixed indentation
\let\hspre\empty
\let\hspost\empty
\newcommand{\NB}{\textbf{NB}}
\newcommand{\Todo}[1]{$\langle$\textbf{To do:}~#1$\rangle$}

\EndFmtInput
\makeatother
%
%
%
%
%
%
% This package provides two environments suitable to take the place
% of hscode, called "plainhscode" and "arrayhscode". 
%
% The plain environment surrounds each code block by vertical space,
% and it uses \abovedisplayskip and \belowdisplayskip to get spacing
% similar to formulas. Note that if these dimensions are changed,
% the spacing around displayed math formulas changes as well.
% All code is indented using \leftskip.
%
% Changed 19.08.2004 to reflect changes in colorcode. Should work with
% CodeGroup.sty.
%
\ReadOnlyOnce{polycode.fmt}%
\makeatletter

\newcommand{\hsnewpar}[1]%
  {{\parskip=0pt\parindent=0pt\par\vskip #1\noindent}}

% can be used, for instance, to redefine the code size, by setting the
% command to \small or something alike
\newcommand{\hscodestyle}{}

% The command \sethscode can be used to switch the code formatting
% behaviour by mapping the hscode environment in the subst directive
% to a new LaTeX environment.

\newcommand{\sethscode}[1]%
  {\expandafter\let\expandafter\hscode\csname #1\endcsname
   \expandafter\let\expandafter\endhscode\csname end#1\endcsname}

% "compatibility" mode restores the non-polycode.fmt layout.

\newenvironment{compathscode}%
  {\par\noindent
   \advance\leftskip\mathindent
   \hscodestyle
   \let\\=\@normalcr
   \let\hspre\(\let\hspost\)%
   \pboxed}%
  {\endpboxed\)%
   \par\noindent
   \ignorespacesafterend}

\newcommand{\compaths}{\sethscode{compathscode}}

% "plain" mode is the proposed default.
% It should now work with \centering.
% This required some changes. The old version
% is still available for reference as oldplainhscode.

\newenvironment{plainhscode}%
  {\hsnewpar\abovedisplayskip
   \advance\leftskip\mathindent
   \hscodestyle
   \let\hspre\(\let\hspost\)%
   \pboxed}%
  {\endpboxed%
   \hsnewpar\belowdisplayskip
   \ignorespacesafterend}

\newenvironment{oldplainhscode}%
  {\hsnewpar\abovedisplayskip
   \advance\leftskip\mathindent
   \hscodestyle
   \let\\=\@normalcr
   \(\pboxed}%
  {\endpboxed\)%
   \hsnewpar\belowdisplayskip
   \ignorespacesafterend}

% Here, we make plainhscode the default environment.

\newcommand{\plainhs}{\sethscode{plainhscode}}
\newcommand{\oldplainhs}{\sethscode{oldplainhscode}}
\plainhs

% The arrayhscode is like plain, but makes use of polytable's
% parray environment which disallows page breaks in code blocks.

\newenvironment{arrayhscode}%
  {\hsnewpar\abovedisplayskip
   \advance\leftskip\mathindent
   \hscodestyle
   \let\\=\@normalcr
   \(\parray}%
  {\endparray\)%
   \hsnewpar\belowdisplayskip
   \ignorespacesafterend}

\newcommand{\arrayhs}{\sethscode{arrayhscode}}

% The mathhscode environment also makes use of polytable's parray 
% environment. It is supposed to be used only inside math mode 
% (I used it to typeset the type rules in my thesis).

\newenvironment{mathhscode}%
  {\parray}{\endparray}

\newcommand{\mathhs}{\sethscode{mathhscode}}

% texths is similar to mathhs, but works in text mode.

\newenvironment{texthscode}%
  {\(\parray}{\endparray\)}

\newcommand{\texths}{\sethscode{texthscode}}

% The framed environment places code in a framed box.

\def\codeframewidth{\arrayrulewidth}
\RequirePackage{calc}

\newenvironment{framedhscode}%
  {\parskip=\abovedisplayskip\par\noindent
   \hscodestyle
   \arrayrulewidth=\codeframewidth
   \tabular{@{}|p{\linewidth-2\arraycolsep-2\arrayrulewidth-2pt}|@{}}%
   \hline\framedhslinecorrect\\{-1.5ex}%
   \let\endoflinesave=\\
   \let\\=\@normalcr
   \(\pboxed}%
  {\endpboxed\)%
   \framedhslinecorrect\endoflinesave{.5ex}\hline
   \endtabular
   \parskip=\belowdisplayskip\par\noindent
   \ignorespacesafterend}

\newcommand{\framedhslinecorrect}[2]%
  {#1[#2]}

\newcommand{\framedhs}{\sethscode{framedhscode}}

% The inlinehscode environment is an experimental environment
% that can be used to typeset displayed code inline.

\newenvironment{inlinehscode}%
  {\(\def\column##1##2{}%
   \let\>\undefined\let\<\undefined\let\\\undefined
   \newcommand\>[1][]{}\newcommand\<[1][]{}\newcommand\\[1][]{}%
   \def\fromto##1##2##3{##3}%
   \def\nextline{}}{\) }%

\newcommand{\inlinehs}{\sethscode{inlinehscode}}

% The joincode environment is a separate environment that
% can be used to surround and thereby connect multiple code
% blocks.

\newenvironment{joincode}%
  {\let\orighscode=\hscode
   \let\origendhscode=\endhscode
   \def\endhscode{\def\hscode{\endgroup\def\@currenvir{hscode}\\}\begingroup}
   %\let\SaveRestoreHook=\empty
   %\let\ColumnHook=\empty
   %\let\resethooks=\empty
   \orighscode\def\hscode{\endgroup\def\@currenvir{hscode}}}%
  {\origendhscode
   \global\let\hscode=\orighscode
   \global\let\endhscode=\origendhscode}%

\makeatother
\EndFmtInput
%


\usepackage{amsthm}
\usepackage{amssymb}
\usepackage{amsmath}

\usepackage{listings}
\lstnewenvironment{code}{\lstset{language=Haskell,basicstyle=\small\ttfamily}}{}

\long\def\ignore#1{}


%\DeclareMathOperator{\mod}{mod}
\DeclareMathOperator{\ord}{ord}
\begin{document}


\ignore{
\begin{hscode}\SaveRestoreHook
\column{B}{@{}>{\hspre}l<{\hspost}@{}}%
\column{E}{@{}>{\hspre}l<{\hspost}@{}}%
\>[B]{}\mathbf{import}\;\Conid{\Conid{Math}.\Conid{Algebra}.\Conid{Group}.PermutationGroup}{}\<[E]%
\\
\>[B]{}\mathbf{import}\;\Conid{\Conid{Math}.\Conid{Core}.Utils}\;((\mathbin{\char94 -})){}\<[E]%
\\
\>[B]{}\mathbf{import}\;\Conid{\Conid{Data}.List}{}\<[E]%
\ColumnHook
\end{hscode}\resethooks
}

Let 

\begin{hscode}\SaveRestoreHook
\column{B}{@{}>{\hspre}l<{\hspost}@{}}%
\column{E}{@{}>{\hspre}l<{\hspost}@{}}%
\>[B]{}\Varid{fish}\mathbin{::}\Conid{Int}\to [\mskip1.5mu \Conid{Permutation}\;\Conid{Int}\mskip1.5mu]{}\<[E]%
\\
\>[B]{}\Varid{fish}\;\Varid{n}\mathrel{=}[\mskip1.5mu \Varid{r1}\;\Varid{n},\Varid{r2}\;\Varid{n}\mskip1.5mu]{}\<[E]%
\\
\>[B]{}\Varid{r1}\mathbin{::}\Conid{Int}\to \Conid{Permutation}\;\Conid{Int}{}\<[E]%
\\
\>[B]{}\Varid{r1}\;\Varid{n}\mathrel{=}\Varid{p}\;[\mskip1.5mu [\mskip1.5mu \mathrm{1}\mathinner{\ldotp\ldotp}\Varid{n}\mskip1.5mu]\mskip1.5mu]{}\<[E]%
\\
\>[B]{}\Varid{r2}\mathbin{::}\Conid{Int}\to \Conid{Permutation}\;\Conid{Int}{}\<[E]%
\\
\>[B]{}\Varid{r2}\;\Varid{n}\mathrel{=}\Varid{p}\;([\mskip1.5mu [\mskip1.5mu \mathrm{2},(\mathrm{2}\mathbin{*}\Varid{n}\mathbin{-}\mathrm{2}),\mathrm{4}\mskip1.5mu]\plus [\mskip1.5mu (\mathrm{2}\mathbin{*}\Varid{n}\mathbin{-}\mathrm{3}),(\mathrm{2}\mathbin{*}\Varid{n}\mathbin{-}\mathrm{4})\mathinner{\ldotp\ldotp}(\Varid{n}\mathbin{+}\mathrm{1})\mskip1.5mu]\mskip1.5mu]){}\<[E]%
\ColumnHook
\end{hscode}\resethooks

Note that this is a subset of $S_{2*n-2}$. 

Let 

\begin{hscode}\SaveRestoreHook
\column{B}{@{}>{\hspre}l<{\hspost}@{}}%
\column{7}{@{}>{\hspre}l<{\hspost}@{}}%
\column{E}{@{}>{\hspre}l<{\hspost}@{}}%
\>[B]{}\Varid{m}\;\Varid{n}\;\Varid{k}\mid \Varid{k}\mathbin{<}\mathrm{0}\mathrel{=}\mathbf{let}\;\Varid{pk}\mathrel{=}\mathbin{-}\Varid{k}\;\mathbf{in}\;((\Varid{r1}\;\Varid{n})\mathbin{\char94 -}\Varid{pk})\mathbin{*}((\Varid{r2}\;\Varid{n})\mathbin{\char94 -}\Varid{pk})\mathbin{*}(\Varid{r1}\;\Varid{n})\mathbin{\uparrow}\Varid{pk}\mathbin{*}(\Varid{r2}\;\Varid{n})\mathbin{\uparrow}\Varid{pk}{}\<[E]%
\\
\>[B]{}\hsindent{7}{}\<[7]%
\>[7]{}\mid \Varid{k}\geq \mathrm{0}\mathrel{=}(\Varid{r1}\;\Varid{n})\mathbin{\uparrow}\Varid{k}\mathbin{*}(\Varid{r2}\;\Varid{n})\mathbin{\uparrow}\Varid{k}\mathbin{*}((\Varid{r1}\;\Varid{n})\mathbin{\char94 -}\Varid{k})\mathbin{*}((\Varid{r2}\;\Varid{n})\mathbin{\char94 -}\Varid{k}){}\<[E]%
\ColumnHook
\end{hscode}\resethooks

Then we find that for any $n\geq 5$, we find that $m\ n\ (\pm 2)$ permutes 4 elements and has order 2. Further, for any $n\geq 6$,  $m\ n\ (\pm 1,3)$ permutes 6 elements and has order 3. It appears that for any $n\geq 4$, we have that $m\ n\ k$ has order 3 and permutes 6 elements if $k \neq \pm 2 \mod n$.

We find that the order of $fish\ 4 =24$ and $\ord(fish\ 5) = 20160=4*7!$. 

For Fish 5, we find that $n_7$ is gotten by



\begin{hscode}\SaveRestoreHook
\column{B}{@{}>{\hspre}l<{\hspost}@{}}%
\column{7}{@{}>{\hspre}l<{\hspost}@{}}%
\column{16}{@{}>{\hspre}l<{\hspost}@{}}%
\column{E}{@{}>{\hspre}l<{\hspost}@{}}%
\>[B]{}\Varid{a}\mathbin{`\Varid{dv}`}\Varid{b}\mathrel{=}(\Varid{b}\mathbin{\Varid{`mod`}}\Varid{a})\equiv \mathrm{0}{}\<[E]%
\\
\>[B]{}\Varid{n7}\mathrel{=}[\mskip1.5mu (\Varid{a},\Varid{b},\Varid{c},\Varid{d})\mid \Varid{a}\leftarrow [\mskip1.5mu \mathrm{0}\mathinner{\ldotp\ldotp}\mathrm{1}\mskip1.5mu],\Varid{b}\leftarrow [\mskip1.5mu \mathrm{0}\mathinner{\ldotp\ldotp}\mathrm{1}\mskip1.5mu],\Varid{c}\leftarrow [\mskip1.5mu \mathrm{0}\mathinner{\ldotp\ldotp}\mathrm{2}\mskip1.5mu],\Varid{d}\leftarrow [\mskip1.5mu \mathrm{0}\mathinner{\ldotp\ldotp}\mathrm{6}\mskip1.5mu]{}\<[E]%
\\
\>[B]{}\hsindent{16}{}\<[16]%
\>[16]{},\mathbf{let}\;\Varid{np}\mathrel{=}\mathrm{7}\mathbin{\uparrow}\Varid{a}\mathbin{*}\mathrm{5}\mathbin{\uparrow}\Varid{b}\mathbin{*}\mathrm{3}\mathbin{\uparrow}\Varid{c}\mathbin{*}\mathrm{2}\mathbin{\uparrow}\Varid{d}{}\<[E]%
\\
\>[B]{}\hsindent{16}{}\<[16]%
\>[16]{},\Varid{np}\mathbin{`\Varid{dv}`}(\mathrm{5}\mathbin{*}\mathrm{3}\mathbin{\uparrow}\mathrm{2}\mathbin{*}\mathrm{2}\mathbin{\uparrow}\mathrm{6}){}\<[E]%
\\
\>[B]{}\hsindent{16}{}\<[16]%
\>[16]{},\Varid{np}\mathbin{\Varid{`mod`}}\mathrm{7}\equiv \mathrm{1}\mskip1.5mu]{}\<[E]%
\\
\>[B]{}\Varid{n5}\mathrel{=}[\mskip1.5mu (\Varid{a},\Varid{b},\Varid{c},\Varid{d})\mid \Varid{a}\leftarrow [\mskip1.5mu \mathrm{0}\mathinner{\ldotp\ldotp}\mathrm{1}\mskip1.5mu],\Varid{b}\leftarrow [\mskip1.5mu \mathrm{0}\mathinner{\ldotp\ldotp}\mathrm{1}\mskip1.5mu],\Varid{c}\leftarrow [\mskip1.5mu \mathrm{0}\mathinner{\ldotp\ldotp}\mathrm{2}\mskip1.5mu],\Varid{d}\leftarrow [\mskip1.5mu \mathrm{0}\mathinner{\ldotp\ldotp}\mathrm{6}\mskip1.5mu]{}\<[E]%
\\
\>[B]{}\hsindent{7}{}\<[7]%
\>[7]{},\mathbf{let}\;\Varid{np}\mathrel{=}\mathrm{7}\mathbin{\uparrow}\Varid{a}\mathbin{*}\mathrm{5}\mathbin{\uparrow}\Varid{b}\mathbin{*}\mathrm{3}\mathbin{\uparrow}\Varid{c}\mathbin{*}\mathrm{2}\mathbin{\uparrow}\Varid{d},((\mathrm{7}\mathbin{*}\mathrm{3}\mathbin{\uparrow}\mathrm{2}\mathbin{*}\mathrm{2}\mathbin{\uparrow}\mathrm{6})\mathbin{\Varid{`mod`}}\Varid{np})\equiv \mathrm{0},\Varid{np}\mathbin{\Varid{`mod`}}\mathrm{5}\equiv \mathrm{1}\mskip1.5mu]{}\<[E]%
\\
\>[B]{}\Varid{n3}\mathrel{=}[\mskip1.5mu (\Varid{a},\Varid{b},\Varid{c},\Varid{d})\mid \Varid{a}\leftarrow [\mskip1.5mu \mathrm{0}\mathinner{\ldotp\ldotp}\mathrm{1}\mskip1.5mu],\Varid{b}\leftarrow [\mskip1.5mu \mathrm{0}\mathinner{\ldotp\ldotp}\mathrm{1}\mskip1.5mu],\Varid{c}\leftarrow [\mskip1.5mu \mathrm{0}\mathinner{\ldotp\ldotp}\mathrm{2}\mskip1.5mu],\Varid{d}\leftarrow [\mskip1.5mu \mathrm{0}\mathinner{\ldotp\ldotp}\mathrm{6}\mskip1.5mu]{}\<[E]%
\\
\>[B]{}\hsindent{7}{}\<[7]%
\>[7]{},\mathbf{let}\;\Varid{np}\mathrel{=}\mathrm{7}\mathbin{\uparrow}\Varid{a}\mathbin{*}\mathrm{5}\mathbin{\uparrow}\Varid{b}\mathbin{*}\mathrm{3}\mathbin{\uparrow}\Varid{c}\mathbin{*}\mathrm{2}\mathbin{\uparrow}\Varid{d},((\mathrm{7}\mathbin{*}\mathrm{5}\mathbin{*}\mathrm{2}\mathbin{\uparrow}\mathrm{6})\mathbin{\Varid{`mod`}}\Varid{np})\equiv \mathrm{0},\Varid{np}\mathbin{\Varid{`mod`}}\mathrm{3}\equiv \mathrm{1}\mskip1.5mu]{}\<[E]%
\\
\>[B]{}\Varid{n2}\mathrel{=}[\mskip1.5mu (\Varid{a},\Varid{b},\Varid{c},\Varid{d})\mid \Varid{a}\leftarrow [\mskip1.5mu \mathrm{0}\mathinner{\ldotp\ldotp}\mathrm{1}\mskip1.5mu],\Varid{b}\leftarrow [\mskip1.5mu \mathrm{0}\mathinner{\ldotp\ldotp}\mathrm{1}\mskip1.5mu],\Varid{c}\leftarrow [\mskip1.5mu \mathrm{0}\mathinner{\ldotp\ldotp}\mathrm{2}\mskip1.5mu],\Varid{d}\leftarrow [\mskip1.5mu \mathrm{0}\mathinner{\ldotp\ldotp}\mathrm{6}\mskip1.5mu]{}\<[E]%
\\
\>[B]{}\hsindent{7}{}\<[7]%
\>[7]{},\mathbf{let}\;\Varid{np}\mathrel{=}\mathrm{7}\mathbin{\uparrow}\Varid{a}\mathbin{*}\mathrm{5}\mathbin{\uparrow}\Varid{b}\mathbin{*}\mathrm{3}\mathbin{\uparrow}\Varid{c}\mathbin{*}\mathrm{2}\mathbin{\uparrow}\Varid{d},((\mathrm{7}\mathbin{*}\mathrm{5}\mathbin{*}\mathrm{3}\mathbin{\uparrow}\mathrm{2})\mathbin{\Varid{`mod`}}\Varid{np})\equiv \mathrm{0},\Varid{np}\mathbin{\Varid{`mod`}}\mathrm{2}\equiv \mathrm{1}\mskip1.5mu]{}\<[E]%
\ColumnHook
\end{hscode}\resethooks


which gives \ensuremath{\Varid{n7}} to be  \ensuremath{[\mskip1.5mu \mathrm{1},\mathrm{8},\mathrm{64},\mathrm{36},\mathrm{288},\mathrm{15},\mathrm{120},\mathrm{960}\mskip1.5mu]}, %\)
\ensuremath{\Varid{n5}} = \ensuremath{[\mskip1.5mu \mathrm{1},\mathrm{16},\mathrm{6},\mathrm{96},\mathrm{36},\mathrm{576},\mathrm{56},\mathrm{21},\mathrm{336},\mathrm{126},\mathrm{2016}\mskip1.5mu]},
\ensuremath{\Varid{n3}} = \ensuremath{[\mskip1.5mu \mathrm{1},\mathrm{4},\mathrm{16},\mathrm{64},\mathrm{10},\mathrm{40},\mathrm{160},\mathrm{7},\mathrm{28},\mathrm{112},\mathrm{448},\mathrm{70},\mathrm{280},\mathrm{1120}\mskip1.5mu]},
\ensuremath{\Varid{n2}} = \ensuremath{[\mskip1.5mu \mathrm{1},\mathrm{3},\mathrm{9},\mathrm{5},\mathrm{15},\mathrm{45},\mathrm{7},\mathrm{21},\mathrm{63},\mathrm{35},\mathrm{105},\mathrm{315}\mskip1.5mu]}

\end{document}
